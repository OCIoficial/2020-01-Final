\documentclass{oci}
\usepackage[utf8]{inputenc}
\usepackage{lipsum}

\title{Bombas}

\begin{document}
\begin{problemDescription}
Hombre Hormiga está en medio de una operación secreta y necesita escapar de la guarida de Chaqueta Amarilla.
Sin embargo su traje tiene una falla y por el momento no le será posible volver a su tamaño normal.
El último obstáculo que debe superar antes de alcanzar su nave es una grilla abierta de dimensiones $M \times N$.
Sin embargo, Chaqueta Amarilla lanzará bombas tóxicas que harán imposible el paso por una casilla.

Hombre Hormiga parte en la posición $(1,1)$ y debe llegar a $(M,N)$, donde lo espera su nave.
Los eventos ocurren por turnos.
En cada turno, CA puede lanzar un número cualesquiera de bombas en casillas diferentes. Cada una de las bombas dejará esa casilla de la grilla impasable debido a los efectos letales de los gases que libera.
Inmediatamente después de eso, HH puede moverse en una dirección cardinal (sur, este, norte, oeste).

¿Existe una forma de que Hombre Hormiga escape, o morirá inevitablemente a manos de Chaqueta Amarilla?
\end{problemDescription}

\begin{inputDescription}
La entrada consiste en varias líneas.
La primera línea tiene dos enteros $M$ y $N$ ($2 \le M, N \le 1\,000$), correspondientes a las dimensiones de la grilla.
La segunda línea contiene un solo entero $K$ ($0 \le K \le MN$), el número de bombas que caerán sobre el terreno.
Las siguientes $K$ líneas describen la caída de las bombas.
Cada una de ellas contiene tres enteros $t_k$ ($1 \le t_k < MN$), $i_k$ ($1 \le i_k \le M$) y $j_k$ ($1 \le j_k \le N$), indicando que una bomba caerá en el $t_k$-ésimo turno en las coordenadas $(i_k, j_k)$.
\end{inputDescription}

\begin{outputDescription}
La salida consiste en una sola línea con un 1 si existe una forma de que Hombre Hormiga escape o 0 si no.
\end{outputDescription}

\begin{scoreDescription}
  \subtask{20}  % Las bombas no caen al principio o bastaría ver que no haya bombas.
  $M = 1$. No hay restricciones adicionales.
  \subtask{30}  % DFS o BFS sirven aquí, el laberinto es estático y flood fill sirve.
  Todas las bombas caen al comienzo, es decir $t_k = 1$ para todos los $k$.
  \subtask{50}
  No hay restricciones adicionales.
\end{scoreDescription}

\begin{sampleDescription}
\sampleIO{sample-1}
\sampleIO{sample-2}
\sampleIO{sample-3}
\end{sampleDescription}

\end{document}

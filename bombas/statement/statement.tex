\documentclass{oci}
\usepackage[utf8]{inputenc}
\usepackage{lipsum}
\usepackage{tikz}
\usetikzlibrary{arrows, decorations.markings}

\title{Bombas}

\begin{document}
\begin{problemDescription}
  Luego de años de búsqueda, el profesor Jones ha logrado finalmente encontrar el cáliz de la vida
  eterna dentro del templo de las mil almas perdidas.
  Pero este no es aún el fin de su aventura.
  Tal como lo presagiaban los escritos, al levantar el cáliz de su pedestal, las mil almas que lo
  resguardaban despertaron de su sueño.
  El templo completo se cae a pedazos y el profesor debe escapar rápidamente con el cáliz o quedará
  atrapado por la eternidad.

  El piso del templo es una grilla de $M\times N$.
  Las casillas están numeradas de norte a sur entre 1 y $M$, y de oeste a este entre 1 y $N$.
  El profesor comienza en la casilla (1, 1) y para escapar debe llegar a la casilla $(M, N)$.
  En condiciones normales se podría caminar libremente sobre la grilla.
  Sin embargo, como todo se cae a pedazos, en cualquier momento puede caer una roca sobre alguna de
  las casillas.

  Los eventos ocurren por turnos, habiendo dos fases en cada uno.
  En la primera fase, un número arbitrario de rocas podrá caer sobre las distintas casillas.
  Si el profesor se encuentra en una de las casillas donde cae una roca, será aplastado por esta y
  lamentablemente no podrá escapar del templo.
  Una vez que una roca cae sobre una casilla, esta quedará bloqueada y el profesor no podrá moverse
  a ella.
  En la segunda fase, el profesor puede moverse desde la casilla actual a una de las cuatro casillas
  adyacentes en dirección norte, sur, este u oeste (siempre y cuando no haya una roca sobre esta).

  En la desesperación, el profesor duda incluso que esta vez le sea posible escapar.
  ?`Puedes ayudar al profesor a saber si podrá escapar o si quedará atrapado para siempre dentro del
  templo?
\end{problemDescription}

\begin{inputDescription}
  La entrada consiste en varias líneas.
  La primera línea contiene dos enteros $M$ y $N$ ($2 \le M, N \le 1\,000$), correspondientes a las
  dimensiones de la grilla.
  La segunda línea contiene un solo entero $K$ ($0 \le K \le M\times N$), correspondiente al número de
  rocas que caerán sobre la grilla.
  Las siguientes $K$ líneas describen la caída de las rocas.
  Cada una contiene tres enteros $t$ ($1 \le t < M\times N$), $i$ ($1 \le i \le M$) y $j$ ($1 \le
  j \le N$), indicando que en la primera fase del $t$-ésimo turno caerá una roca en las coordenadas $(i,
  j)$.
  Se garantiza que nunca caerán rocas sobre las casillas (1, 1) y $(M, N)$.
\end{inputDescription}

\begin{outputDescription}
  La salida de contener una sola línea con un 1 si existe una forma de que el profesor escape o un 0
  si no.
\end{outputDescription}

\begin{scoreDescription}
  \subtask{20}  % Las bombas no caen al principio o bastaría ver que no haya bombas.
  Se probarán varios casos donde $M = 1$ y sin restricciones adicionales.
  \subtask{30}  % DFS o BFS sirven aquí, el laberinto es estático y flood fill sirve.
  Se probarán varios casos donde todas las bombas caen al comienzo, es decir $t = 1$ para todas las rocas.
  \subtask{50}
  Se probarán varios casos sin restricciones adicionales.
\end{scoreDescription}

\begin{sampleDescription}
  \sampleIO{sample-1}
  \sampleIO{sample-2}
  \sampleIO{sample-3}
\end{sampleDescription}
\end{document}

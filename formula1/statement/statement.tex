\documentclass{oci}
\usepackage[utf8]{inputenc}
\usepackage{lipsum}

\title{Fórmula 1}

\begin{document}
\begin{problemDescription}
  Todos los corredores están situados en la pista, listos para partir.
  La bandera verde indica el inicio y los competidores arrancan a máxima velocidad.

  En una carrera de la fórmula 1 los autos pueden alcanzar fácilmente velocidades por sobre los 300 km/h.
  A estas velocidades es difícil para los espectadores percibir todos los detalles.
  Afortunadamente, una pantalla muestra en cada momento las estadísticas de la carrera.

  Entre los datos mostrados en la pantalla uno es la diferencia de tiempo entre un corredor y el
  corredor inmediatamente antes que él.
  Este dato es muy útil pues ilustra qué tan alejado está un competidor del siguiente.
  Lamentablemente, este dato solo sirve para comparar corredores en posiciones contiguas y durante
  la carrera a muchos analistas les gustaría poder comparar cualquier par de competidores.

  De forma específica, la pantalla muestra la diferencia en tiempo en la que dos competidores
  terminaron la vuelta anterior.
  Es decir, si al terminar la vuelta anterior, el corredor en la posición $(i-1)$-ésima cruzó la
  meta en el tiempo $t_{i-1}$ y el corredor en la posición $i$-ésima lo hizo en el tiempo $t_{i}$,
  la pantalla mostrará junto al corredor en la posición $i$-ésima el valor $t_{i} - t_{i-1}$.
  Por ejemplo, si el corredor que va tercero cruzó la meta luego de 300 segundos
  comenzada la carrera y el corredor que va cuarto la cruzó 305 segundos después de comenzada la
  carrera, la pantalla mostrará un valor igual a $305-300=5$ para el corredor en la cuarta posición.
  Dado que el corredor en la primera posición no tiene nadie antes que él, la pantalla siempre mostrará
  el valor 0 para este corredor.

  Notar que las posiciones de los competidores pueden variar constantemente y los valores en la
  pantalla solo consideran las posiciones en que los competidores terminaron la última vuelta.
  Notar además que los valores solo tienen sentido una vez que todos los competidores han completado
  la vuelta anterior.

  Dadas las diferencias en tiempo entre los competidores en posiciones contiguas, tu tarea es
  responder diferentes consultas indicando la diferencia en tiempo entre dos competidores en
  posiciones arbitrarias.
\end{problemDescription}

\begin{inputDescription}
  La entrada comienza con dos enteros $N$ y $Q$ indicando respectivamente la cantidad de
  competidores en la carrera y la cantidad de consultas que debes responder.

  La siguiente línea contiene $N$ enteros \textbf{mayores o iguales que cero}.
  El $i$-ésimo entero corresponde a la diferencia en tiempo entre el corredor en la posición
  $i$-ésima y el corredor en la posición $(i-1)$-ésima, tal como se describe en el enunciado.
  El primer entero, por corresponder al corredor que va primero en la carrera, siempre será 0.
  
  Finalmente, siguen $Q$ líneas cada una describiendo una consulta.
  Cada una de estas líneas contiene dos enteros $a$ y $b$ indicando que debes
  responder la diferencia en tiempo entre los competidores en las posiciones $a$ y $b$.

  En todos los casos de prueba siempre se cumplirán las siguientes restricciones:
  \begin{itemize}
  \item $2 \leq N \leq 1\,000\,000$
  \item $1 \leq Q \leq 1\,000\,000$
  \item $1 \leq a < b \leq N$
  \end{itemize}
\end{inputDescription}

\begin{outputDescription}
  La salida debe contener $Q$ líneas cada una conteniendo un entero \textbf{mayor o igual que cero}.
  La $j$-ésima línea debe contener la respuesta a la pregunta $j$-ésima.
  Se garantiza que la respuesta a las consultas siempre será menor que $10^9$.
\end{outputDescription}

\begin{scoreDescription}
  \subtask{40}
  Se probarán varios casos en que $N \leq 1\,000$ y $Q \leq 1\,000$.
  \subtask{60}
  Se probarán varios casos sin restricciones adicionales.
\end{scoreDescription}

\begin{sampleDescription}
  \sampleIO{sample-1}
  \sampleIO{sample-2}
\end{sampleDescription}

\end{document}
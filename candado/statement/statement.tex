\documentclass{oci}
\usepackage[utf8]{inputenc}
\usepackage{lipsum}

\title{El candado del basurero}

\begin{document}
\begin{problemDescription}
  Recientemente, Emilio se cambió a un nuevo departamento.
  Emilio es una persona muy meticulosa y ha gastado gran parte de su tiempo en arreglar su
  nuevo hogar para que todo funcione como a él le gusta.
  Luego de algunos meses trabajando, ya ha logrado dejar todas las cosas como a él le gustan.
  Bueno, casi todo.
  Todavía hay algo que le molesta.

  El nuevo edificio donde se encuentra su departamento tiene un contenedor de basura donde Emilio
  debe llevar su basura cada vez que el basurero dentro de su departamento se llena.
  Por orden del administrador del edificio, el basurero debe siempre permanecer cerrado con un
  candado.
  Esto irrita mucho a Emilio pues cada vez que lleva la basura tiene que sacar y luego volver a
  poner el candado.
  Lo cuál, según él, toma mucho tiempo y es especialmente molesto cuando está lloviendo.

  Para abrir el candado, se debe ingresar la combinación de 4 dígitos rotando los discos
  giratorios.
  Emilio lo hace rotando uno a uno los discos siempre una posición a la vez.
  Es decir, rotar uno de los discos a una de las posiciones contiguas, hacia arriba o hacia abajo, es
  considerado por Emilio un movimiento.

  Dependiendo de la posición inicial de cada disco puede ser conveniente rotarlo en uno u otro
  sentido.
  Por ejemplo, si el primer disco se encuentra en la posición 8 y hay que llevarlo a la posición 0 es
  conveniente rotarlo hacia arriba, primero a la posición 9 y luego a la 0.
  De esta forma solo se deben hacer dos movimientos.
  Por otro lado, rotarlo hacia abajo toma 8 movimientos.

  Emilio piensa que si de antemano sabe en que dirección rotar cada disco podrá abrir el candado en
  la menor cantidad de movimientos posibles ahorrando así invaluable tiempo.
\end{problemDescription}

\begin{inputDescription}
  La entrada contiene dos líneas cada una conteniendo cuatro enteros entre 0 y 9.
  La primera línea corresponde a las posiciones de los cilindros en la que comienza el candado.
  La segunda línea corresponde a la combinación correcta del candado, es decir, las posiciones a las
  que hay que llevar los cilindros.
\end{inputDescription}

\begin{outputDescription}
  La salid debe contener un único entero correspondiente a la mínima cantidad de movimientos en los
  que es posible llegar a la combinación correcta.
\end{outputDescription}

\begin{scoreDescription}
  Este problema no contiene subtareas se dará un puntaje proporcional a la cantidad de casos de
  prueba correctos siendo 100 el máximo puntaje.
\end{scoreDescription}

\begin{sampleDescription}
\sampleIO{sample-1}
\sampleIO{sample-2}
\sampleIO{sample-3}
\end{sampleDescription}

\end{document}
